\documentclass[conference]{IEEEtran}
\IEEEoverridecommandlockouts
% The preceding line is only needed to identify funding in the first footnote. If that is unneeded, please comment it out.
\usepackage{cite}
\usepackage{amsmath,amssymb,amsfonts}
\usepackage{algorithmic}
\usepackage{graphicx}
\usepackage{textcomp}
\usepackage{xcolor}
\usepackage{hyperref}
\usepackage{listings}
\usepackage{float}
\usepackage{todonotes}

\def\BibTeX{{\rm B\kern-.05em{\sc i\kern-.025em b}\kern-.08em
    T\kern-.1667em\lower.7ex\hbox{E}\kern-.125emX}}

\newcommand{\arrowright}{$\,\to\,$}
    
\begin{document}

\title{App-controlled LEGO 3-DoF robotic arm\\
{\footnotesize Project of lecture ''Mobile Computing'' (winter term 2018/2019)}
}

\author{
\IEEEauthorblockN{Christoph Ulrich}
\IEEEauthorblockA{%\textit{dept. name of organization (of Aff.)} \\
\textit{HTWG Konstanz}\\
Constance, Germany\\
christoph.ulrich@htwg-konstanz.de}
\and
\IEEEauthorblockN{Benjamin Schaefer}
\IEEEauthorblockA{%\textit{dept. name of organization (of Aff.)} \\
\textit{HTWG Konstanz}\\
Constance, Germany\\
benjamin.schaefer@htwg-konstanz.de}
}

\maketitle

\begin{abstract}
it's reasonable to write this after all other sections of this paper have been completed
\end{abstract}

%\begin{IEEEkeywords}
%component, formatting, style, styling, insert
%\end{IEEEkeywords}\

\section{Introduction / Motivation}
\begin{itemize}
	\item control of a robotic arm is a fundamental task in robotics - easy hands-on experience for everybody to this fundamental robotic application with this paper and the created low cost LEGO 3-DoF robot arm  (incl. instructions)
	\item application and hardware could be used in basic lecture "Grundlagen der mobilen Robotik" to better understand robot kinematics, ROS and a bit of perception
	\item recycling of old and unused hardware of the robotics lab at the HTWG Konstanz
	\item typical industrial applications to control a robotic arm run on more powerful hardware and often offer a complicated and - for beginners - confusing GUI, \todo[author=Christoph, inline]{insert example(s)} so we developed an easy to use mobile application for Android platforms
	\item ROS because widely used, very modular/extensible and basic framework which almost every student starting with robotics has to get in touch with
\end{itemize}
\par

\todo[author=Christoph, inline]{new paragraph - description of background and main ''problem''}
\begin{itemize}
	\item which platform to choose for controlling the  arm and driving the motors (Arduino, Raspberry, etc.) - should consume as little energy as possible,  should be flexible and portable
	\todo[author=Christoph, inline]{note that in \ref{sec:platform}}
	\item another aspect \arrowright how should the application on the mobile device communication with the controlling device \arrowright BT, WiFi, (lost of steering commands due to radio lacks etc.)
	\item arm construction \arrowright not too many components, not too heavy so that the motors are able to drive the arm even with more than one joint and with gripper load (to demonstrate gripping)
	\item app \arrowright 
\end{itemize}
Constructing a robot arm generally leads to some difficulties.

\section{State of the Art}

\todo[author=Beide, inline]{find two or three example applications/hardware compontents, analyze and compare them, also look at mobile application programming techniques used in these works}
\begin{itemize}
	\item https://www.hackster.io/slantconcepts/control-arduino-robot-arm-with-android-app-1c0d96
	\item https://www.instructables.com/id/Robot-Arm-Arduino-App/
	\item https://www.kuka.com/en-us/products/robotics-systems/software/application-software/kuka-hrc-guide-app
\end{itemize}

\section{Proposed Approach}\label{sec:approach}

\subsection{Requirements Engineering}\label{sec:requirements}
\todo[author=Beide, inline]{what should the arm be able to achieve in the end? How should the app look like and which functions does it have to provide?}

\subsection{Platform Decision}\label{sec:platform}
\todo[author=Beide, inline]{Raspberry, Arduino, NXT, EV3, others?}

\subsection{Arm Construction}\label{sec:construction}
\begin{figure}[bt] 
	\centering
	\includegraphics[width=\textwidth/2]{img/arm_full_circles}
	\caption[caption]{Figure of the Lego Technic arm. The lower joint is marked in red and the upper joint in blue}}
	\label{fig:arm_full}
\end{figure}
The arm, which can be seen in Figure 1, was designed entirely from Lego Technic. For this purpose, 2 NXT bricks were used. In addition, 4 motors and a touch sensor were used. An NXT was used to control the two arm joints, as well as the gripper. Since the NXTs only have three ports for motors, a second NXT was needed to rotate the base.\\
The servomotors have a built-in rotation sensor with an accuracy of 1 degree. Since these servomotors only have a torque of about 12 N.cm, a translation has been built into the wrist joint so that the motors can move the arm. For this purpose, a translation of 1/42 was used at the lower joint (red circle). The engine's ratio at the upper joint (blue circle) was 2/25.\\
In order to keep the arm as light as possible and to relieve the engines, sensors on the arm itself were completely dispensed with. However, since the engines have to be initialized, a touch sensor has been installed on the base to initialize the first arm articulated. The initialization of the second joint was solved by the torque of the motors, as there is less weight of the arm.

\subsection{Algorithms}\label{sec:algorithms}
\begin{figure}[bt] 
	\centering
	\includegraphics[width=\textwidth/2]{img/calibration.png}
	\caption[caption]{Activity diagram of the calibration process}}
\label{fig:calibration}
\end{figure}
The following describes the calibration procedure, which is shown in Figure 2. Since the servomotors can only be controlled by the effort and have no encoder, the starting position of the motors is unknown. Therefore, these must be calibrated. A touch sensor is used for the lower joint. The lower motor rotates until the touch sensor responds to a touch. Then the upper arm is calibrated. Here, the engines rotate with the least force so long, with which the motors can just turn until the joint is in the final position and the engines can not turn, because the effort is too great. In order not to burden the motors unnecessarily, the calibration process is terminated as soon as a certain delta has been exceeded. Finally, the same procedure is repeated for the gripper.
\todo[author=Christoph, inline]{implementation of forward and backward kinematics }

\subsection{App Development}\label{sec:development}
\todo[author=Christoph, inline]{communication/process description, ROS, navigation strategy, ... }

\subsection{Expected Results}\label{sec:expectedresults}
\todo[author=Christoph, inline]{speed, accuracy, ...}

\section{Results}

\section{Conclusion}

\section{Further Work}
\begin{itemize}
	\item 3D graphics in App
	\item EV3
	\item algorithms
	\item external better sensors to optimize
	\item ...
\end{itemize}


\begin{thebibliography}{00}
\bibitem{onlPrimitives} 
Khronos - OpenGL: Primitives - Triangle Primitives,
\\\texttt{https://www.khronos.org/opengl/wiki/Primitive}
\end{thebibliography}

\listoftodos

\end{document}